\synctex=1

\gdef\lecturenumber{a1}
\gdef\subsectioncounters{1}
\input preamble.tex


\begin{document}
\date{due: Thursday, 2022--07--09, 11:59 pm}
\Chapter{\vspace*{-2ex}Assignment 1}

~\\[-7ex]

{\noindent You} may \textbf{not} work in a group on this assignment.

\medskip

{\noindent ID number(s)}: 20311734\\[1ex]
{\noindent Name(s)}: Zhimin Zhao

\vspace*{1ex}

\section{Grammars}

Suppose we have the following grammar.

\begin{grammar}
  integers
  &
  $n$
  &$\in \mathbb{Z}$
\\
  variables
  &
  $x$, $y$, $z$
\\
  systems
  &
  $s$ 
  &\bnfas&
  $
  \keyword{start}~z~\keyword{in}~s
  \bnfaltBRK
  \keyword{stop}
  \bnfaltBRK
  \keyword{get}~y~z
  \bnfaltBRK
  \keyword{set}~x~\keyword{to}~y
  \bnfaltBRK
  s~;~s
  $
\end{grammar}

\textbf{Part (a).} According to the above grammar, which of the following strings can be produced from the
nonterminal (meta-variable) $s$?  

\begin{minipage}[t]{0.47\linewidth}
\begin{enumerate}
\item $\keyword{stop}~;~\keyword{stop}~;~\keyword{stop}$
\item $\keyword{set}~x, y~\keyword{to}~z~;~\keyword{get}~z~x~;~\keyword{stop}$
\item $\keyword{get}~5~z$
\end{enumerate}
\end{minipage}
~
\begin{minipage}[t]{0.47\linewidth}
\begin{enumerate}
  \setcounter{enumi}{3}
\item $\keyword{start}~y~\keyword{in}~\keyword{get}~z~\keyword{stop}$
\item $\keyword{start}~z~\keyword{in}~\keyword{start}~z~\keyword{in}~\keyword{get}~z~z$
\item $\keyword{set}~x~\keyword{to}~\keyword{stop}~;~\keyword{get}~x~y$
\end{enumerate}
\end{minipage}

\medskip

List their numbers, or put a checkmark next to those that can be produced:

\bigskip
1,5
\bigskip

\textbf{Part (b).} 
Below, I have drawn one of two possible
parse trees for the system
\[
\keyword{start}~z~\keyword{in}~\keyword{set}~y~\keyword{to}~z~;~\keyword{get}~y~x~
\]
Draw the \emph{other} possible parse tree.
\begin{verbatim}
            start_in_                                       ;
                 /   \                                  /       \
                z     ;                            start_in_    get
                     / \                            /   \       /  \
               set_to_  get                        z  set_to_  y    x
               / \      / \                            /   \
              y  z     y   x                          y     z
\end{verbatim}
\clearpage

\textbf{Part (c).}
Lisp-style syntax, like $\plusexp{1}{\plusexp{2}{3}}$, eliminates the possibility of multiple parse trees
for the same string.
Rewrite the above grammar of $s$ to use Lisp-style syntax: every production should look like
\texttt{(\keyword{word}~$\cdots$)}, where \keyword{word} is unique to that production
(\eg don't write two productions that both start with \keyword{get})
and ``$\cdots$'' contains only meta-variables ($n$, $x$, $y$, $z$, $s$).

I have given you the first production; for the last production, my intent was that $s~;~s$
represents a sequence of systems, so the word \keyword{seq} would be a reasonable choice.

\begin{grammar}
  integers
  &
  $n$
  &$\in \mathbb{Z}$
\\
  variables
  &
  $x$, $y$, $z$
\\
  systems
  &
  $s$ 
  &\bnfas&
  $
  \texttt(\keyword{start}~x~s\texttt)
  \vspace{1ex}
  \bnfaltBRK
% If you are doing your assignment in TeX, enter your solution on a new line after this one.
  \keyword{stop}
  \vspace{1ex}
  \bnfaltBRK
  \texttt(\keyword{get}~y~z\texttt)
  \vspace{1ex}
  \bnfaltBRK
  \texttt(\keyword{set}~x~y\texttt)
  \vspace{1ex}
  \bnfaltBRK
  \texttt(\keyword{seq}~s~s\texttt)
  ~~~~
  $
\end{grammar}


\section{Grammars as inductive definitions}

Following the example at the beginning of lec2 \S4 for arithmetic expressions, 
write an inductive definition that corresponds to the original grammar of systems
(\emph{not} your grammar with Lisp-like syntax).
I have given you the first part.

\textbf{Hint}: The grammar allows the string $\keyword{get}~x~y~;~\keyword{stop}$.
Make sure that your definition also allows that string.

\begin{enumerate}[(a)]
\item If $x$ is a variable and $s$ is a system,
  then $\keyword{start}~x~\keyword{in}~s$
  is a system.\vspace{-0.6ex}
\item If $s$ is a system and $s'$ is a system,
  then $s~;~s'$
  is a system.\vspace{-0.6ex}
\item If $x$ is a variable and $y$ is a variable,
  then $\keyword{get}~x~y$
  is a system.\vspace{-0.6ex}
\item If $x$ is a variable and $y$ is a variable,
  then $\keyword{set}~x~\keyword{to}~y$
  is a system.\vspace{-0.6ex}
\item $\keyword{stop}$ is a system.
\vspace{6ex}
\end{enumerate}
\vspace{6ex}



Now, for the Lisp-style grammar, write the part of the inductive definition that
corresponds to the first production, $\texttt(\keyword{start}~x~s\texttt)$.

\begin{enumerate}[(a)]
\item If $x$ is a variable and $s$ is a system, then $\texttt{(}\keyword{start}~x~s\texttt{)}$ is a system.
\end{enumerate}
 \vspace{6ex}


\clearpage
\section{``In logic, there are no morals.''}
\newcommand{\downodd}{\down_\textsf{\upshape\selectfont odd}}
\newcommand{\odddown}{\mathrel{{}_\textsf{\upshape\selectfont odd}{\down}}} %_\textsf{\upshape\selectfont0}

Recognizing when things are wrong is just as important as recognizing when
things are right.

We define a judgment, $e \downodd n$, that resembles our big-step semantics
but ``likes odd numbers''.

\medskip
\judgbox{e \downodd n}{expression $e$ evaluates to $n$, but likes odd numbers}
\vspace*{-1.2ex}
    \begin{mathpar}
      \Infer{evalodd-const}
          {}
          {n \downodd 7}
      \and
      \Infer{evalodd-add}
          {
            e_1 \downodd n_1
            \\
            e_2 \downodd n_2
          }
          {\plusexp{e_1}{e_2} \downodd 2n_1 + 2n_2 + 1}
    \end{mathpar}

Prove the following conjecture.  You may choose whether to induct on the expression
or the derivation, and whether to do case analysis on the expression or the derivation;
for this proof, either approach can work.

\textbf{After ``By structural induction on'', state what you are inducting on.}

\textbf{After ``Induction hypothesis:'', write the appropriate induction hypothesis.}

You will need to ``case-split'' on something.
\textbf{After ``Consider cases of :'', write what you are considering cases of}
(for example: the form of the expression $e$, or the rule concluding $\Dee$,
or something else of your choosing).

Write out every step in detail.  Use additional pages if necessary.

\subsection*{Proof of ``Always odd''}

\begin{conjecture}[Always odd] ~\\
  For all expressions $e$ and derivations $\Dee$ such that $\Dee$ derives $e \downodd n$,\\
  there exists $m \in \mathbb{Z}$ such that $n = 2m + 1$.
\end{conjecture}
\begin{proof}
  By structural induction on $e$
  \vspace{1ex} 
  \\
  \textbf{Induction hypothesis}:
  For all expressions $e'\prec{e}$ and derivations $\Dee'$ such that $\Dee'$ derives $e'~\downodd~n$, there exists $m\in\mathbb{Z}$ such that $n = 2m + 1$.
  \vspace{2ex}

  Consider cases of $e$

  \begin{itemize}
  \item Case: $e = n$.\\
  $n = 7$\hspace{10mm}By inversion\\
  $n = 2*3 + 1$\hspace{10mm}By arithmetic\\
  Let $m = 3$.\\
  $n = 2m + 1$\\
  
  \item Case: $e = \plusexp{e_1}{e_2}$.\\
  For all expressions $e_1, e_2\prec{e}$ and derivations $\Dee_1$, $\Dee_2$ such that $\Dee_1$ derives $e_1~\downodd~n_1$, $\Dee_2$ derives $e_2~\downodd~n_2$, there exists $m_1, m_2\in\mathbb{Z}$ such that $n_1 = 2m_1 + 1$, $n_2 = 2m_2 + 1$.\\
  $\plusexp{e_1}{e_2}\downodd 2n_1+2n_2+1$\hspace{10mm}By evalodd-add\\
  $n = 2n_1+2n_2+1$\hspace{10mm}By inversion\\
  Let $m = n_1 + n_2$.\\
  $n = 2m + 1$\\
  \end{itemize}
    \vfill
\end{proof}

\subsection*{Soundness and completeness}

Now consider yet another judgment, which also likes odd numbers:
to evaluate an integer expression, it must be odd.

\medskip
\judgbox{e \odddown v}{expression $e$ evaluates to value $v$, but only works for odd numbers}
    \begin{mathpar}
      \Infer{oddeval-const}
          {}
          {2n + 1 \odddown 2n + 1}
      \and
      \Infer{oddeval-add}
          {
            e_1 \odddown n_1
            \\
            e_2 \odddown n_2
          }
          {\plusexp{e_1}{e_2} \odddown n_1 + n_2}
    \end{mathpar}

We now have three different definitions of big-step evaluation: the ``real'' definition $e \down v$,
and two ``joke'' definitions, $e \downodd v$ and $e \odddown v$.
Let's compare these definitions in the framework of soundness and completeness.
Since we have (I hope) some faith in the validity of $e \down v$, we will consider that
to be our ``ground truth'', and compare the joke definitions with respect to $e \down v$.  We repeat those rules here:

\medskip
\judgbox{e \down v}{expression $e$ evaluates to value $v$}
\vspace*{-1.2ex}
    \begin{mathpar}
      \Infer{eval-const}
          {}
          {n \down n}
      \and
      \Infer{eval-add}
          {
            e_1 \down n_1
            \\
            e_2 \down n_2
          }
          {\plusexp{e_1}{e_2} \down n_1 + n_2}
    \end{mathpar}

Each of the following questions asks you to either state \emph{and prove} the appropriate theorem, or give a counterexample.


\begin{enumerate}[(a)]
\item Is the theory $e \downodd v$ sound with respect to $e \down v$?

  If it is sound, state \emph{and prove} the appropriate theorem.

  \checkmark If it is not sound, give a counterexample ($e$ and $v$ such that
  $e \downodd v$ is derivable, but $e \down v$ is not derivable).
  
  Let $e=10$.
  
  \begin{llproof}
  \Pf{}{}{e\downodd 7}{By evalodd-const}
  \Pf{}{}{e\down 10}{By eval-const}
  \Pf{}{}{7\neq 10}{By arithmetic}
  \end{llproof}
  
  Thus it is not sound.

\item Is the theory $e \downodd v$ complete with respect to $e \down v$?

  If it is complete, state \emph{and prove} the appropriate theorem.

  \checkmark If it is not complete, give a counterexample ($e$ and $v$ such that
  $e \down v$ is derivable, but $e \downodd v$ is not derivable).
  
  Let $e=10$.
  
  \begin{llproof}
  \Pf{}{}{e\down 10}{By eval-const}
  \Pf{}{}{e\downodd 7}{By evalodd-const}
  \Pf{}{}{10\neq 7}{By arithmetic}
  \end{llproof}
  
  Thus it is not sound.

\item Is the theory $e \odddown v$ sound with respect to $e \down v$?

  \checkmark If it is sound, state \emph{and prove} the appropriate theorem.

  If it is not sound, give a counterexample ($e$ and $v$ such that
  $e \odddown v$ is derivable, but $e \down v$ is not derivable).
  
  \textbf{Problem Statement}: Theory $e \odddown v$ is sound with respect to $e \down v$
  
  \textbf{Proof}: By structural induction on $e$
  
  \textbf{Induction hypothesis}: For all $e'\prec{e}$ expressions and derivations $\Dee'$ such that $\Dee'$ derives $e'\odddown v'$, then $e'\down v'$.
  \vspace{2ex}
  
  Consider cases of $e$
  \begin{enumerate}
  \item Case: $e = 2n + 1$.\\
  \begin{llproof}
  \Pf{}{}{e\odddown 2n + 1}{By oddeval-const}
  \Pf{}{}{e\down 2n + 1}{By eval-const}
  \end{llproof}
  
  \item Case: $e = \plusexp{e_1}{e_2}$.\\
  \begin{llproof}
  \Pf{}{}{e_1\odddown n_1}{By assumption}
  \Pf{}{}{e_1\down n_1}{By IH}
  \Pf{}{}{e_2\odddown n_2}{By assumption}
  \Pf{}{}{e_2\down n_2}{By IH}
  \Pf{}{}{e\odddown n_1 + n_2}{By oddeval-add}
  \Pf{}{}{e\down n_1 + n_2}{By eval-add}
  \end{llproof}
  \end{enumerate}
\end{enumerate}

A counterexample fits on a line, and a proof doesn't, so I can't give you an appropriate
amount of space without giving away parts of the answers.
Therefore, please write your answers on the next page.

If you need to write one or more proofs, I suggest following the
language of Conjecture 3.1: ``For all \dots, \dots'' and the form of that proof:
say what you are inducting on, state the induction hypothesis, and give all necessary
detail.

\end{document}

% Local Variables:
% TeX-master: "a1"
% End:
